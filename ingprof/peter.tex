\documentclass[12pt,a4paper]{article}
\usepackage[utf8]{inputenc}
\usepackage{fancyhdr}
%\usepackage{datetime2}

\pagestyle{fancy}
\fancyhf{}

\lhead{pettu298}
\rhead{2016-05-14} % yyyy-mm-dd
\setlength{\headheight}{15pt}

\cfoot{\thepage}

\begin{document}

\begin{center}
    \Huge
    \textbf{Ingenjörsproffesionalism}

    \vspace{0.3cm}
    \Large
    Peter Tullstedt
    
    \vspace{0.7cm}
    \textbf{Livskarriär}
\end{center}


\section{Beskrivning}

Efter min civilingenjörsexamen kommer jag att jobba med mjukvaruutveckling i åtminstone några år. Det skulle väl inte direkt förvåna mig om jag jobbar med det ännu längre, då jag i alla fall just nu tycker att det är väldigt kul med mjukvaruutveckling.

\section{Känslor, tankar och reaktioner}

För mig är det alltid viktigare att ha fungerande vänskapsförhållanden än något annat. Efter det så är det också väldigt viktigt för mig att ha någon sysselsättning, då jag inte riktigt mår bra av att inte ha något strukturerat att göra från dag till dag. Det kan till exempel vara ett jobb, men skulle lika gärna kunna vara någon hobby eller annan aktivitet som gör att jag har något som behöver göras.\\

Saker som lön och förmåner spelar mindre roll för mig än att jag trivs med ett jobb.

\section{Värdering}

Att ha en framtidsvision är väldigt bra, framförallt i situationer då man inte känner sig helt bekväm med sin nuvarande situation, t.ex. när studierna går dåligt. Då är det bra att komma ihåg sitt mål med studierna.

\section{Analys}

Jag tror att det är viktigt för mig att sysselsätta mig för att inte hamna i en situation där jag vänjer mig vid att inte behöva göra saker. När jag är van vid att inte behöva göra saker så har jag väldigt svårt för att börja göra saker igen, när det faktiskt är viktigt.\\

Jag tror att jag alltid kommer att prioritera att vara nöjd med mitt liv utanför arbetet över själva arbetet, men samtidigt självklart försöka få ut så mycket av det arbete jag har.\\

Jag tror att det viktigaste jag behöver komma ihåg att tänka på i framtiden är att alltid evaluera om jag är nöjd med situationen jag är i. 

\section{Generella slutsatser}

Lycka, framgång och karriär hänger inte nödvändigtvis ihop för alla personer, men de gör nog det för många. För mig så känner jag att lycka oftast leder till framgång, i alla fall inom vissa saker. När jag mår bra så brukar jag kunna vara väsentligt mycket mer produktiv och producerar högkvalitativa produkter.\\

Jag tror att för alla så är det viktigt att ta reda på vilka saker som är viktiga för en själv, och fokusera mest på att se till att de sakerna går bra.

\section{Personliga handlingsplaner}

Jag tänker göra karriär i livet.


\end{document}